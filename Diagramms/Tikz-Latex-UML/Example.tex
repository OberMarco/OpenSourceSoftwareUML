%%%%%%%%%%%%%%%%%%%%%%%%%%%%%%%%%%%%%%%%%%%%%%%%%%%%%%%%%%%%%%%%%%%%%%%%%%%%%%%%%
%Beamer-Präsentationsvorlage
%%%%%%%%%%%%%%%%%%%%%%%%%%%%%%%%%%%%%%%%%%%%%%%%%%%%%%%%%%%%%%%%%%%%%%%%%%%%%%%%%

\documentclass[xcolor=dvipsnames]{beamer}


%\usepackage{xcolor}
\usetheme{default}
\usepackage{qrcode}
% deutsche Silbentrennung
\usepackage[ngerman]{babel}
% Deutsche Umlaute
\usepackage[utf8]{inputenc}
% Einbinden von Bildern
\usepackage{graphicx}
% Unterstützung von Hyperlinks
\usepackage{hyperref}

% Mathematische Pakete
\usepackage{amsmath}
\usepackage{amssymb}
\usepackage{amsfonts}
\usepackage{amsthm}
\usepackage{latexsym}
\usepackage{colortbl}
\usepackage{listings}

% Abweichung von Vorlage
% um Graphen zu plotten
\usepackage{tikz}
\usetikzlibrary{shapes, arrows, automata, chains, positioning, shadows, matrix}



\usepackage{tikzscale}
\usepackage{tikz-uml}\usepackage{tabularx}


\makeatletter
\newcommand{\trickbeamer}{%
	\advance\beamer@slideinframe by-1%
}%
\makeatother



\def \variableXYZ{Ueberschrift1}
\def \variableXYZ{Ueberschrift1}
\newcommand{\neueFolie}[1]{\frametitle{#1}\section{#1}}


\definecolor{rgray}{rgb}{0.95,0.95,0.95}
\definecolor{darkgreen}{rgb}{0.0,0.5,0.0}
\lstset{
	language=[LaTeX]TeX, 
	captionpos=b, 
	frame=lines, 
	basicstyle=\ttfamily, 
	keywordstyle=\color{blue}, 
	commentstyle=\color{darkgreen}, 
	stringstyle=\color{red}, 
	%numbers=left, 
	%numberstyle=\tiny, 
	%numbersep=5pt,
	breaklines=true, 
	showstringspaces=false, 
	emph={double,bool,int,unsigned,char,true,false,void},
	emphstyle=\color{blue},
	emph={Assert,Test},
	emphstyle=\color{red},
	emph={[2]\using,\#define,\#ifdef,\#endif}, 
	emphstyle={[2]\color{blue}},
	backgroundcolor=\color{rgray}
}

\input{Gestaltung/gestaltung_oth}
%%%%%%%%%%%%%%%%%%%%%%%%%%%%%%%%%%%%%%%%%%%%%%%%%%%%%%%%%%%%%%%%%%%%%%%%%%%%%%%%%
%Hier werden die Daten für die erste Folie eingetragen
%%%%%%%%%%%%%%%%%%%%%%%%%%%%%%%%%%%%%%%%%%%%%%%%%%%%%%%%%%%%%%%%%%%%%%%%%%%%%%%%%
\title{ UML Tools}
\subtitle{Open-Source Software for the Working Scientist \\}
\institute{Master of Applied Research\\ Hochschule Ansbach\\}
\author{Marco Obermeier}
\date{\today}
%\date{12. Januar 2022}

\begin{document} 
	\begin{frame}
		\frametitle{~}
		\maketitle 
	\end{frame}
	%%%%%%%%%%%%%%%%%%%%%%%%%%%%%%%%%%%%%%%%%%%%%%%%%%%%%%%%%%%%%%%%%%%%%%%%%%%%%%%%%
	%Einleitung
	%%%%%%%%%%%%%%%%%%%%%%%%%%%%%%%%%%%%%%%%%%%%%%%%%%%%%%%%%%%%%%%%%%%%%%%%%%%%%%%%%
	%\begin{frame}
	%	\frametitle{Inhaltsverzeichnis}
	%	\tableofcontents
	%\end{frame}
	%%%%%%%%%%%%%%%%%%%%%%%%%%%%%%%%%%%%%%%%%%%%%%%%%%%%%%%%%%%%%%%%%%%%%%%%%%%%%%%%%
	%Einleitung
	%%%%%%%%%%%%%%%%%%%%%%%%%%%%%%%%%%%%%%%%%%%%%%%%%%%%%%%%%%%%%%%%%%%%%%%%%%%%%%%%%
\begin{frame}
	\frametitle{Beispiel}
			\begin{minipage}{1.0\textwidth}	
				\resizebox{\columnwidth}{!}{%
					\begin{tikzpicture}
						
					
						\umlbasiccomponent[x=0, y=0,fill=yellow!20]{A} 
						\umlbasiccomponent[x=4, y=0,fill=yellow!20]{B}
						\umlbasiccomponent[x=8, y=0,fill=yellow!20]{C}
						
			
						\umlassemblyconnector[interface=interface, anchor1=0,  first arm=1.5cm]{A}{B}
						\umlassemblyconnector[interface=interface2, anchor1=0,  first arm=1.5cm]{B}{C}
					

						
						
						
						%\umlVHVassemblyconnector[interface=3D-Beschleunigung, anchor1=0,  first arm=3.5cm]{MySQL}{Datenkollektor}
						%\umlassemblyconnector[interface=Winkelgeschwidigkeit, anchor1=0]{Flugsteuerung}{Gyrosensor}
						%\umlHVHassemblyconnector[interface=PWM-Signal, anchor1=-62,  first arm=3.5cm]{Flugsteuerung}{Motoren}
		
						%\umlVHVassemblyconnector[interface=Lagewinkel]{Regelung}{Sensorauswertung} 
						%\umlVHVassemblyconnector[interface=Motordrehzahl]{Motorsteuerung}{Regelung} 
						
						
					\end{tikzpicture}
				}
			\end{minipage}
\end{frame}



	
	
\end{document}