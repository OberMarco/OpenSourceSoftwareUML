\documentclass{article}

% Language setting
% Replace `english' with e.g. `spanish' to change the document language
\usepackage[english]{babel}

% Set page size and margins
% Replace `letterpaper' with`a4paper' for UK/EU standard size
\usepackage[letterpaper,top=2cm,bottom=2cm,left=3cm,right=3cm,marginparwidth=1.75cm]{geometry}

% Useful packages
\usepackage{amsmath}
\usepackage{graphicx}
\usepackage[colorlinks=true, allcolors=blue]{hyperref}

\title{Projectdescription for the course \\ \textit{Open-Source Software for the Working Scientist}}
\author{Marco Obermeier}

\begin{document}
	\maketitle
	
	\section{Introduction}
	
	To implement structured programs it is necessary to design UML(Unified Modeling Language)-diagrams. 
	This is especially important in a scientific context, because it enables the scientific teams to better combine and unify solution strategies.
	In scientific projects, the project budget is usually small. For this reason, expensive software products such as \href{https://www.visual-paradigm.com/}{\textit{visual paradigm}} are not needed to create the diagrams. Therefore, different open source products for the creation of UML diagrams will be investigated.
	
	
	\section{Projectdescription}
	
	Based on the creation of 1-2 UML diagrams, differences, advantages and disadvantages of open source solutions for UML diagrams will be worked out.
	The created UML diagram will continue to be used in the research work of the Masters of applied Research. 
	The focus of the diagrams lies in the abstract and clear representation of software and software designs.
	
	\subsection{Software products}
	
	In this project, the following open source software are investigated and compared:
	\begin{itemize}
		\item \href{https://plantuml.com/de/}{\textit{PlantUML}}
		\item \href{https://perso.ensta-paris.fr/~kielbasi/tikzuml/index.php}{\textit{Integrated Latex UML - TikZ-UML}}
		\item \href{https://www.modelio.org/}{\textit{Modelio}}
		\item ... ?
	\end{itemize}
	
	
	\subsection{Investigation criteria}

	\begin{itemize}
		\item Time required for creation
		\item Time required for modification
		\item General visual impression
		\item ... ? 
	\end{itemize}
	
	
\end{document}